\documentclass[11pt]{beamer}
\usetheme{Boadilla}
\usepackage[utf8]{inputenc}
\usepackage[czech]{babel}
\usepackage[T1]{fontenc}
\usepackage{amsmath}
\usepackage{amsfonts}
\usepackage{amssymb}
\usepackage{tikz}
\usetikzlibrary{automata,arrows,calc,positioning}
\usepackage{graphicx}
\usepackage{pgfpages}
\usepackage{minted}

\author{Jan Tušil}

\title{A Generic Framework for Symbolic Execution}
%\title{A Generic Framework for Symbolic Execution: a Coinductive Approach}
%\setbeamercovered{transparent} 
%\setbeamertemplate{navigation symbols}{} 
%\logo{} 
%\institute{} 
%\date{} 
%\subject{} 

%\setbeameroption{show notes}
%\setbeameroption{show notes on second screen=right}

\begin{document}

\begin{frame}
\titlepage
\end{frame}

\begin{frame}
\tableofcontents
\end{frame}

\section{Intro}

\begin{frame}[fragile=singleslide]{MojeIntro}

\begin{minted}[linenos=true]{cpp}
int x,y;
x = get();
y = -x;
y = -y;
assert(x == y);
\end{minted}

%\pause 

Může assert selhat?

\end{frame}

\begin{frame}{Operační sémantika}
\begin{equation*}
OpSem : Program \rightarrow TransitionSystem
\end{equation*}
\pause
\begin{figure}
\begin{tikzpicture}%[roundnode/.style={circle, draw=black!60, fill=black!5, very thick, minimum size=7mm}]
%Nodes
%\node[]      (fst)                              {\textbullet};
%\node[]      (midcircle) [above right = of fst] {\textbullet};
\node[]      (fst)                              {\textbullet};
\node[]      (n11) [above right = of fst] {\textbullet};
\node[]      (n12) [right = of n11] {\textbullet};
\node[]      (n13) [right = of n12] {\textbullet};
\node[]      (n21) [below = of n11] {\textbullet};
\node[]      (n22) [right = of n21] {\textbullet};
\node[]      (n23) [right = of n22] {\textbullet};
\node[]      (n31) [below = of n21] {\textbullet};
\node[]      (n32) [right = of n31] {\textbullet};
\node[]      (n33) [right = of n32] {\textbullet};
\node[]      (n01) [above = of n11] {\textbullet};
\node[]      (n02) [right = of n01] {\textbullet};
\node[]      (n03) [right = of n02] {\textbullet};
\node[]      (ok) [right = of n23] {ok};
\node[]      (fail) [right = of n33] {fail};

\draw[-latex] (fst) to[] node[left]{$1$}(n01);
\draw[-latex] (n01) to[] node[]{}(n02);
\draw[-latex] (n02) to[] node[]{}(n03);
\draw[-latex] (n03) to[] node[]{}(ok);
\draw[-latex] (fst) to[] node[above]{$2$}(n11);
\draw[-latex] (n11) to[] node[]{}(n12);
\draw[-latex] (n12) to[] node[]{}(n13);
\draw[-latex] (n13) to[] node[]{}(ok);
\draw[-latex] (fst) to[] node[above]{$2^{31}$}(n21);
\draw[-latex] (n21) to[] node[]{}(n22);
\draw[-latex] (n22) to[] node[]{}(n23);
\draw[-latex] (n23) to[] node[]{}(ok);
\draw[-latex] (fst) to[] node[right]{$- 2^{31}$}(n31);
\draw[-latex] (n31) to[] node[]{}(n32);
\draw[-latex] (n32) to[] node[]{}(n33);
\draw[-latex] (n33) to[] node[]{}(fail);
\path (n11) -- node[auto=false]{\vdots} (n21);
\path (n12) -- node[auto=false]{\vdots} (n22);
\path (n13) -- node[auto=false]{\vdots} (n23);
%\draw[-latex] (fst) to[] node[]{}(n11);

%\node[]      (midcircle) [above right = of fst] {\textbullet};
%\node[roundnode]      (uppercircle)       [above=of midcircle] {1};
%\node[roundnode]      (rightcircle)       [right=of midcircle] {4};
%\node[roundnode]      (lowercircle)       [below=of midcircle] {3};
%
%%Edges out of 1
%\draw[->] (uppercircle.east) to[bend left=20] node[above right]{-}(rightcircle.north);
%\draw[->] (uppercircle.-135) to[bend right=40] node[right]{-}(lowercircle.135);
%\draw[->] (uppercircle.135) to[out=135,in=180] ($(uppercircle) +(0,3em)$)node[above]{+} to[out=0, in =45] (uppercircle.45);
%
%%2 Edges out of 2
%\draw[->] (midcircle.east) --node[above]{+} (rightcircle.west);
%\draw[->] (midcircle.80) to[out=80,in=135] ($(midcircle) +(2em,2em)$)node[right]{+}  to[out=-45, in =10] (midcircle.10);
%\draw[->] (midcircle.south) -- (lowercircle.north);
%\draw[->] (midcircle.north) -- (uppercircle.south);
%
%%Edges out of 3
%\draw[->] (lowercircle.150) to[bend left=80]node[left]{+}(uppercircle.-150);
%\draw[->] (lowercircle.east) to[bend right=20]node[below right]{-}(rightcircle.south);
%\draw[->] (lowercircle.-135)  to[out=-135,in=180] ($(lowercircle) +(0,-3em)$)node[below]{+} to[out=0, in =-45](lowercircle.-45);
%

%Edges out of 4
%\draw[->] (rightcircle.-135) to[bend left=20] node[below]{-} (midcircle.-45);

\end{tikzpicture}
\end{figure}
\end{frame}

\section{Logics}


\begin{frame}{FOL}
\begin{equation}
\phi ::= \top \mid p(t_1,\ldots,t_n) \mid \neg \phi \mid \phi \land \phi \mid \left( \exists X \right) \phi
\end{equation}
\end{frame}

\begin{frame}{ML}

Signature ML: $ 123 $

%\pause
\begin{equation}
\varphi ::= \pi \mid \top \mid p(t_1,\ldots,t_n) \mid \neg \varphi \mid \varphi \land \varphi \mid \left( \exists V \right) \varphi
\end{equation}
\end{frame}

\end{document}