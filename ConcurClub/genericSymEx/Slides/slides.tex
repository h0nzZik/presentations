\documentclass[11pt]{beamer}
\usetheme{Boadilla}
\usepackage[utf8]{inputenc}
\usepackage[czech]{babel}
\usepackage[T1]{fontenc}
\usepackage{amsmath}
\usepackage{amsfonts}
\usepackage{amssymb}
\usepackage{tikz}
\usetikzlibrary{automata,arrows,calc,positioning}
\usepackage{graphicx}
\usepackage[tight]{k}
\usepackage{listings}
\usepackage{pgfpages}
%\usepackage{minted}

\author{Jan Tušil}

\title{A Generic Framework for Symbolic Execution}
%\title{A Generic Framework for Symbolic Execution: a Coinductive Approach}
%\setbeamercovered{transparent} 
%\setbeamertemplate{navigation symbols}{} 
%\logo{} 
%\institute{} 
%\date{} 
%\subject{} 

%\setbeameroption{show notes}
%\setbeameroption{show notes on second screen=right}

\newtheorem{principle}{Princip}
\newtheorem{dfn}{Definice}
\newtheorem{pozn}{Poznámka}


\DeclareMathOperator{\SortOf}{SortOf}


\begin{document}

\begin{frame}
\titlepage
\end{frame}

\begin{frame}
\tableofcontents
\end{frame}

\section{Intro}

\begin{frame}[fragile=singleslide]{MojeIntro}

\begin{lstlisting}
int x,y;
x = get();
y = -x;
y = -y;
assert(x == y);
\end{lstlisting}
%\pause 

Může assert selhat?

\end{frame}

\begin{frame}{Operační sémantika}
\begin{equation*}
OpSem : Program \rightarrow TransitionSystem
\end{equation*}
\pause
\begin{figure}
\begin{tikzpicture}%[roundnode/.style={circle, draw=black!60, fill=black!5, very thick, minimum size=7mm}]
%Nodes
%\node[]      (fst)                              {\textbullet};
%\node[]      (midcircle) [above right = of fst] {\textbullet};
\node[]      (fst)                              {\textbullet};
\node[]      (n11) [above right = of fst] {\textbullet};
\node[]      (n12) [right = of n11] {\textbullet};
\node[]      (n13) [right = of n12] {\textbullet};
\node[]      (n21) [below = of n11] {\textbullet};
\node[]      (n22) [right = of n21] {\textbullet};
\node[]      (n23) [right = of n22] {\textbullet};
\node[]      (n31) [below = of n21] {\textbullet};
\node[]      (n32) [right = of n31] {\textbullet};
\node[]      (n33) [right = of n32] {\textbullet};
\node[]      (n01) [above = of n11] {\textbullet};
\node[]      (n02) [right = of n01] {\textbullet};
\node[]      (n03) [right = of n02] {\textbullet};
\node[]      (ok) [right = of n23] {ok};
\node[]      (fail) [right = of n33] {fail};

\draw[-latex] (fst) to[] node[left]{$1$}(n01);
\draw[-latex] (n01) to[] node[]{}(n02);
\draw[-latex] (n02) to[] node[]{}(n03);
\draw[-latex] (n03) to[] node[]{}(ok);
\draw[-latex] (fst) to[] node[above]{$2$}(n11);
\draw[-latex] (n11) to[] node[]{}(n12);
\draw[-latex] (n12) to[] node[]{}(n13);
\draw[-latex] (n13) to[] node[]{}(ok);
\draw[-latex] (fst) to[] node[above]{$2^{31} - 1$}(n21);
\draw[-latex] (n21) to[] node[]{}(n22);
\draw[-latex] (n22) to[] node[]{}(n23);
\draw[-latex] (n23) to[] node[]{}(ok);
\draw[-latex] (fst) to[] node[right]{$- 2^{31}$}(n31);
\draw[-latex] (n31) to[] node[]{}(n32);
\draw[-latex] (n32) to[] node[]{}(n33);
\draw[-latex] (n33) to[] node[]{}(fail);
\path (n11) -- node[auto=false]{\vdots} (n21);
\path (n12) -- node[auto=false]{\vdots} (n22);
\path (n13) -- node[auto=false]{\vdots} (n23);
%\draw[-latex] (fst) to[] node[]{}(n11);

%\node[]      (midcircle) [above right = of fst] {\textbullet};
%\node[roundnode]      (uppercircle)       [above=of midcircle] {1};
%\node[roundnode]      (rightcircle)       [right=of midcircle] {4};
%\node[roundnode]      (lowercircle)       [below=of midcircle] {3};
%
%%Edges out of 1
%\draw[->] (uppercircle.east) to[bend left=20] node[above right]{-}(rightcircle.north);
%\draw[->] (uppercircle.-135) to[bend right=40] node[right]{-}(lowercircle.135);
%\draw[->] (uppercircle.135) to[out=135,in=180] ($(uppercircle) +(0,3em)$)node[above]{+} to[out=0, in =45] (uppercircle.45);
%
%%2 Edges out of 2
%\draw[->] (midcircle.east) --node[above]{+} (rightcircle.west);
%\draw[->] (midcircle.80) to[out=80,in=135] ($(midcircle) +(2em,2em)$)node[right]{+}  to[out=-45, in =10] (midcircle.10);
%\draw[->] (midcircle.south) -- (lowercircle.north);
%\draw[->] (midcircle.north) -- (uppercircle.south);
%
%%Edges out of 3
%\draw[->] (lowercircle.150) to[bend left=80]node[left]{+}(uppercircle.-150);
%\draw[->] (lowercircle.east) to[bend right=20]node[below right]{-}(rightcircle.south);
%\draw[->] (lowercircle.-135)  to[out=-135,in=180] ($(lowercircle) +(0,-3em)$)node[below]{+} to[out=0, in =-45](lowercircle.-45);
%

%Edges out of 4
%\draw[->] (rightcircle.-135) to[bend left=20] node[below]{-} (midcircle.-45);

\end{tikzpicture}
\end{figure}
\end{frame}

\begin{frame}{Konfigurace}
\begin{multline*}
\mall{blue}{k}{x=get(); \kra y=-x; \kra y=-y; \kra assert(x == y);} \mall{green}{env}{x=0,y=0}
\end{multline*}
\pause
Ukázka
\end{frame}

\begin{frame}{Symbolická Konfigurace}
\begin{multline*}
\mall{blue}{k}{y=-x; \kra y=-y; \kra assert(x == y);} \mall{green}{env}{x=X,y=0}
\end{multline*}
\pause
Ukázka
\end{frame}

\begin{frame}{Symbolická exekuce}

\begin{principle}[Pokrytí]
Každému (potenciálně nekonečnému) konkrétnímu běhu odpovídá nějaký symbolický běh.
\end{principle}

\pause

\begin{principle}[Přesnost]
Každému konečnému symbolickému běhu odpovídá nějaký konkrétní běh.
\end{principle}

\pause Nekonečné běhy - koindukce

\end{frame}

\section{Jazyk, logika, sémantika}

\subsection{Logika konfigurací}

\begin{frame}[fragile]{Signatura}

\begin{pozn}[Sortování]
Říkáme-li o množině X, že je Y-sortovaná, máme tím na mysli, že existuje funkce $\SortOf : X \rightarrow Y $.
\end{pozn}

\pause

\begin{dfn}
\textit{Vícedruhová algebraická signatura} je tvořena množinou sortů $S$ spolu s $S^* \times S$-sortovanou množinou $\Sigma$ funkčních symbolů. Symbol $T_{\Sigma}$ označuje $\Sigma$-algebru uzavřených termů, $T_{\Sigma , s}$ množinu termů sortu $s$, $T_{\Sigma}\left( Var \right)$ volnou $\Sigma$-algebru termů s proměnnými.
\end{dfn}

\pause

\begin{figure}
\begin{lstlisting}
Plant ::= favouriteFood(Animal)
Animal ::= mother(Animal) | father(Animal)
\end{lstlisting}
\end{figure}

\pause

Příklad.

\end{frame}

% Jako příklad - různé termy, i otevřené

\begin{frame}{Vícedruhová FOL}

\begin{dfn}
Signatura provořádové logiky $\left( \Sigma, \Pi \right)$ je tvořena algebraickou signaturou $\Sigma$ a $S^*$-sortovanou množinou predikátových symbolů $\Pi$.
\end{dfn}

\pause

\begin{dfn}[Formule]
Množina formulí nad signaturou $\left( \Sigma, \Pi \right)$ je definována:
\begin{equation*}
\phi ::= \top \mid p(t_1,\ldots,t_n) \mid \neg \phi \mid \phi \land \phi \mid \left( \exists X \right) \phi
\end{equation*}
kde $p$ označuje predikátové symboly, $X$ podmnožiny proměnných a $t_i$ $\Sigma$-termy s volnými proměnnými.
\end{dfn}

\pause
Příklad (StaršíNež).
\pause
Předpokládáme predikát rovnosti.

\end{frame}

\newcommand{\folSignature}{$\left( \Sigma, \Pi \right)$}

%TODO konvence oznacovani sortovych podmnozin
\begin{frame}{Modely}
Analogie s realizacemi jazyků v MA007.
\begin{dfn}
Model signatury \folSignature je $\Sigma$-algebra $\mathcal{T}$ spolu s realizací $\mathcal{T}_p \subseteq \mathcal{T}_{s_1} \times \cdots \times \mathcal{T}_{s_n} $ pro každý predikátový symbol $p \in \Pi_{s_1 \ldots s_n}$.
\end{dfn}

\pause
Zvířecí příklad.
\pause

\begin{dfn}[Relace splnitelnosti]
Pro \folSignature -formuli $\phi$, \folSignature -model $\mathcal{T}$ a valuaci $\rho : Var \rightarrow \mathcal{T}$ definujeme relaci splnitelnosti $\rho \models \phi$ jako obvykle. (Valuaci $\rho$ přirozeně rozšiřujeme na morfismus $\Sigma$-algeber $\rho : T_{\Sigma}\left( Var \right) \rightarrow \mathcal{T}$.)
\end{dfn}
\pause
Zkusme to na tabuli.
\pause
Příklad.


%
%Splnitelnost pro fixovaný model.
%\pause
%
%\begin{pozn}[Substituce]
%Substituce je funkce $\sigma : Var \rightarrow T_{\Sigma}\left( Var \right)$
%\end{pozn}

\end{frame}
%
%%TODO rovnost


\begin{frame}{Matching Logic - logika konfigurací}

%\pause
\begin{equation}
\varphi ::= \pi \mid \top \mid p(t_1,\ldots,t_n) \mid \neg \varphi \mid \varphi \land \varphi \mid \left( \exists V \right) \varphi
\end{equation}
\end{frame}

%TODO sorty pro identifikátory apod.

\subsection{Logika běhů}

\end{document}